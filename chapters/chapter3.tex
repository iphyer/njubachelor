\chapter{模板使用说明}

\section{模板的宗旨}

我们已经知道,\LaTeX{} 通过命令和环境实现各种复杂的排版效果。
在导言区,我们可以通过调入宏包,
利用调入宏包提供的各种设置命令以及自定义各种命令,
来定制文档的排版效果。
如果导言区有很多内容,有时我们把它写在另外一个~.tex 文件中,
利用 \verb|\input{文件}| 的方式调入到导言区中。
这样做的好处是精简文档代码,尽量把内容与自定义设置分开。
在书写某些大型的文档时,
我们也可以使用 \verb|\input{章节文件.tex}| 的方式对文档进行拆分。
编译时只需要调入正在编辑的章节内容,而不必编译整份文档,节约编译的时间。
当然,利用命令 \verb|\input{章节文件.tex}| 来拆分文档有相当的局限性,以后再作说明。

对于比较复杂的模板,更好的方法是将自定义的内容制作成宏包或者文类。
文类可以提供选项,这样一个文类可以根据不同的选项提供不同的排版。
文类中可以方便地重定义某些 \LaTeX{} 的内部命令,来改变排版效果。
更重要的,对于我这样的菜鸟,用文类写模板看起来似乎更加专业一些。
好吧,其实除了看起来更专业我真的没觉得用文类能比 \verb|\input{}| 优越太多。

当然,毕业论文模板另一个相当重要的功能就是提供其他用户一些命令,
以方便地排版出封面、摘要等内容。
用模板排版封面、摘要等内容,无论质量还是方便性,与 Word 相比都有巨大的优势。

\section{开始}

使用本模板,方法十分简单。
我们已经知道怎样开始一个标准的 \LaTeX{} 文档,
使用本模板,只需要把调用的文类替换成 \verb|NJUbachelor| 即可。

{\small\vspace{-0.2em}%
\begin{verbatim}
              \documentclass[参数]{NJUbachelor}
                导言区
              \begin{document}
                正文
              \end{document}
\end{verbatim}}


\section{参数}

你可以直接调用模板而不给出任何参数,缺省选项是为最通用的情况设置。
当然,你也可以根据自己的需求输入参数,来改变缺省设置。
可选的参数有
\begin{description}
  \item[thesis/design] 封面页显示“本科毕业论文”或者“本科毕业设计”。
    其中 \texttt{thesis} 为缺省设置。
  \item[oneside/twoside] 设置单面或者双面打印。
    双面打印会区分单双页,页眉和左右页边距都会根据单双页自动调整。
    双面打印时还会在必要的地方插入空白页。
    其中 \texttt{oneside} 为缺省设置。
  \item[pageheader/nopageheader] 设置是否打印页眉。
    \texttt{nopageheader} 为缺省设置。
  \item[chapter/nochapter] 设置是否支持章 \verb|\chapter{}| 层次。
    不使用章层次更符合学校要求。
    不使用章层次时,最高一级标题是 \verb|\section{节标题}|。
    节标题套用学校章的要求使用四号黑体。
    (简单的说,就是学校要求中的章模板中由 \verb|\section{}| 命令给出)
    毕竟本科学位论文还没有长到要用 \LaTeX{} 中的章层次的地步。
    如果用章层次,则每一章都会另起一页开始。
    使用章层次将无法在正文生成中文标题,
    中文标题的格式三号宋体加粗被套用给章,
    而节的格式仍然是四号黑体。
    其中 \texttt{nochapter} 为缺省设置。
  \item[openright/openany] 如果有章层次,
    设置章层次可在任意页新起一页或者必须在右手(奇数)页新起一页。
    其中 \texttt{openright} 为缺省设置。
  \item[longtitle/shorttitle/manualtitle]
    设置标题页和中文摘要纸的中文标题排版效果。
    长标题可支持自动换行,左对齐显示;
    短标题适合单行标题,可在横线上居中显示;
    手动标题可以利用 \verb|\mtitle| 命令提供手动换行支持,
    每一行标题都可以在横线上居中显示。
    其中 \texttt{longtitle} 为缺省设置。
  \item[shortdepspe/longdepspe]
    你可以根据自己院系和专业名的总长度进行选择。
    选择 shortdepspe 的排版效果和学校给出的中文摘要页是一样的;
    如果院系专业名太长,可以选择 longdepspe 选项。
    其中 \texttt{shortdepspe} 为缺省设置。
  \item[defaultmath/mathptm/mathtxf] 你可以根据自己的喜好选择数学字体。
    其中 \texttt{defaultmath} 为缺省设置。
\end{description}


\section{基本信息输入}

论文基本信息的输入必须在生成标题页和摘要页之前(这似乎很显然)。
建议你在导言区输入论文的基本信息,
以便你的论文信息可以正确写入生成 pdf 的文件属性。
根据学校所给标题页和中英文摘要纸的要求,
本模板定义了以下命令来给出论文的基本信息。
关于基本信息的输入,可以参考 start.tex,这里就不赘述了。


\section{正文前内容}

正文前内容包括封面,中英文摘要纸和目录。
首先,封面页由命令 $$ \verb|\makectitlepage| $$ 生成。
它将自动把导言区输入的论文基本信息填写到封面相应的空缺处。
然后分别是中英文标题页,它们分别由中英文摘要环境给出

{\vspace{-0.2em}%
\begin{verbatim}
            \begin{cabstract}      \begin{eabstract}
              中文摘要。              English abstract.
            \end{cabstract}        \end{eabstract}
\end{verbatim}}

\vspace{-0.2em}\noindent
你只需要在环境中写入摘要的内容,
摘要纸的其它空缺同样会依据导言区给出的基本信息自动进行填充。
最后是目录。
学校规定非正文内容要用大写罗马数字作页码。
所以首先利用命令 \verb|\pagenumbering{Roman}| 规定页眉格式。
同时目录页是不需要页眉的,
因此使用 \verb|\thispagestyle{plain}| 规定页面样式为无页眉。
如果使用的是 nochapter 模式,
目录标题作为节层次,摆放的位置会略微偏高,影响美观,
可以试着插入一条 \verb|\vspace*{0em}| 命令。
为了让生成 pdf 的标签带有目录这一条目,
再插入命令 \verb|\pdfbookmark[0]{目录}{contents}|。
最后,是正式的目录生成命令 $$\verb|\tableofcontents|$$
至此,正文前的内容就全部输入完成了。


\section{正文书写}

正文需要另起一页并用阿拉伯数字编号页码,
需要用到命令 \verb|\newpage| 和 \verb|\pagenumbering{arabic}|。
正文第一页一般同样无页眉,因此再跟一条命令 \verb|\thispagestyle{plain}|。
接着,生成正文标题 $$\verb|\makectitle|$$
需要注意的是,chapter 模式下我强行把这一命令设置为空,
即使使用也无法得到输出。
同时,这一模式下我把标题的格式赋给了章标题。
因此 chapter 模式并不很符合学校所给文件的要求。

接着,就可以书写论文正文了。
根据个人的使用经验,为了方便,这里还给出了三个常用的自定义命令。
\begin{description}
  \item[\texttt{\textbackslash{}diff}]
    定义了微分算符,用法前面已经提到。
  \item[\texttt{\textbackslash{}scite\{citekey\}}]
    用以提供参考文献引用的上标效果。
  \item[\texttt{\textbackslash{}unit\{UNIT\}}]
    定义了单位命令,如输入 \verb|5\unit{cm^{2}}|,
    得到 5\unit{cm^{2}},单位自动写成正体,并与数字空开一点距离。
\end{description}

\section{正文后内容}

正文后内容也需要用 \verb|\newpage| 另起一页,
并用 \verb|\pagenumbering{Roman}| 设置页码为大写罗马数字。
