\chapter{模板使用说明}

\section{模板的宗旨}

我们已经知道,\LaTeX{} 通过命令和环境实现各种复杂的排版效果。
在导言区,我们可以通过调入宏包,
利用调入宏包提供的各种设置命令以及自定义各种命令,
来定制文档的排版效果。
如果导言区有很多内容,有时我们把它写在另外一个~.tex 文件中,
利用 \verb|\input{文件}| 的方式调入到导言区中。
这样做的好处是精简文档代码,尽量把内容与自定义设置分开。
在书写某些大型的文档时,
我们也可以使用 \verb|\input{章节文件.tex}| 的方式对文档进行拆分。
编译时只需要调入正在编辑的章节内容,而不必编译整份文档,节约编译的时间。
当然,利用命令 \verb|\input{章节文件.tex}| 来拆分文档有相当的局限性,以后再作说明。

对于比较复杂的模板,更好的方法是将自定义的内容制作成宏包或者文类。
文类可以提供选项,这样一个文类可以根据不同的选项提供不同的排版。
文类中可以方便地重定义某些 \LaTeX{} 的内部命令,来改变排版效果。
更重要的,对于我这样的菜鸟,用文类写模板看起来似乎更加专业一些。
好吧,其实除了看起来更专业我真的没觉得用文类能比 \verb|\input{}| 优越太多。

当然,毕业论文模板另一个相当重要的功能就是提供其他用户一些命令,
以方便地排版出封面、摘要等内容。
用模板排版封面、摘要等内容,无论质量还是方便性,与 Word 相比都有巨大的优势。

\section{开始}

使用本模板,方法十分简单。
我们已经知道怎样开始一个标准的 \LaTeX{} 文档,
使用本模板,只需要把调用的文类替换成 \verb|NJUbachelor| 即可。

{\small\vspace{-0.2em}%
\begin{verbatim}
              \documentclass[参数]{NJUbachelor}
                导言区
              \begin{document}
                正文
              \end{document}
\end{verbatim}}


\section{参数}

你可以直接调用模板而不给出任何参数,缺省选项是为最通用的情况设置。
当然,你也可以根据自己的需求输入参数,来改变缺省设置。
可选的参数有
\begin{description}
  \item[thesis/design] 封面页显示“本科毕业论文”或者“本科毕业设计”。
  \item[oneside/twoside] 设置单面或者双面打印。
    双面打印会区分单双页,页眉和左右页边距都会根据单双页自动调整。
    双面打印时还会在必要的地方插入空白页。
  \item[pageheader/nopageheader] 设置是否打印页眉。
  \item[chapter/nochapter] 设置是否支持章 \verb|\chapter{}| 层次。
    不使用章层次更符合学校要求。
    不使用章层次时,最高一级标题是 \verb|\section{节标题}|。
    节标题套用学校章的要求使用四号黑体。
    (简单的说,就是学校要求中的章模板中由 \verb|\section{}| 命令给出)
    毕竟本科学位论文还没有长到要用 \LaTeX{} 中的章层次的地步。
    如果用章层次,则每一章都会另起一页开始。
    使用章层次将无法在正文生成中文标题,
    中文标题的格式三号宋体加粗被套用给章,
    而节的格式仍然是四号黑体。
  \item[openright/openany] 如果有章层次,
    设置章层次可在任意页新起一页或者必须在右手(奇数)页新起一页。
  \item[longtitle/shorttitle/manualtitle]
    设置标题页和中文摘要纸的中文标题排版效果。
    长标题可支持自动换行,左对齐显示;
    短标题适合单行标题,可在横线上居中显示;
    手动标题可以利用 \verb|\mtitle| 命令提供手动换行支持,
    每一行标题都可以在横线上居中显示。
  \item[shortdepspe/longdepspe] 
    你可以根据自己院系和专业名的总长度进行选择。
    选择 shortdepspe 的排版效果和学校给出的中文摘要页是一样的;
    如果院系专业名太长,可以选择 longdepspe 选项。
  \item[defaultmath/mathptm/mathtxf] 你可以根据自己的喜好选择数学字体。
\end{description}
